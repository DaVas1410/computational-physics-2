\documentclass[12pt,a4paper]{extarticle}

\date{\today}

\usepackage{amssymb}
\usepackage{amsmath}
\usepackage[a4paper, margin=0.4in]{geometry}
\usepackage[colorlinks]{hyperref}
\usepackage[hiresbb]{graphicx}
\usepackage[round]{natbib}   % omit 'round' option if you prefer square brackets
\usepackage{wrapfig}
\usepackage[T1]{fontenc}
\usepackage{enumitem}
\usepackage{afterpage}

\newcommand\blankpage{%
    \null
    \thispagestyle{empty}%
    \addtocounter{page}{-1}%
    \newpage}

\renewcommand{\figurename}{Figura}
\usepackage{caption}

\hypersetup{
    linkcolor=red,          % color of internal links (change box color with linkbordercolor)
    citecolor=blue,        % color of links to bibliography
    filecolor=magenta,      % color of file links
    urlcolor=blue           % color of external links
}

\usepackage{listings}
\usepackage{xcolor}


% Define Python syntax highlighting style
\lstdefinestyle{mystyle}{
    language=Python,
    basicstyle=\ttfamily\footnotesize,
    keywordstyle=\color{blue},
    commentstyle=\color{gray},
    stringstyle=\color{red},
    numbers=left,
    numberstyle=\tiny\color{gray},
    stepnumber=1,
    breaklines=true,
    frame=single,
    captionpos=b
}

\lstset{style=mystyle}


\begin{document}

%\maketitle
\thispagestyle{empty}

\newpage
\pagestyle{empty}

\begin{center}\section*{Quiz 4 - Computational Physics II}\end{center}
\vspace{0.2cm}
{\bf NAME: \verb|_________________________________________________|  SCORE:} \\
%{\bf Subject:} Quantum Mechanics I \\
{\bf Date:} Thursday 29 May 2025 (17h00) \hspace{0.25cm} {\bf Duration:} 45 minutes\\
  \hspace{0.25cm} {\bf Credits:} 20 points (4 questions) \hspace{0.25cm}  {\bf Type of evaluation:} LAB\\

\vspace{-0.25cm}

\noindent {\bf Provide \underline{concise answers} to the following items:}\\

\begin{enumerate}

\item {\bf (4 points) Partial differential equations (PDEs) in Fourier space}\\
(a) Write down the 1D heat equation and the 1D one-way wave equation in Fourier space.\\
(b) Explain the difference between diffusion and advection processes.\vspace{8cm}


\item {\bf (6 points) Numerical Stability}\\
Explain 3 different methods by which we can determine the stability of a numerical scheme.\newpage

\item {\bf (5 points) Finite-difference methods for PDEs}\\
Write down the 3D Poisson equation and its central-difference approximation including errors.\vspace{9cm}


\item {\bf (5 points) Boundary Conditions for advection problems}\\
Consider the advection equation, $u_t + c\,u_x = 0$ (with $c>0$) on the domain $x \in [0, L]$ with $N_x$ physical grid points plus 2 ghost zones (one at each end). The array \lstinline{u} has size $N_x+2$, and the physical points have indices $1$ to $N_x$. The CFL number is defined as $\text{CFL} = c\,\Delta t / \Delta x$. The upwind scheme for interior points is: \lstinline{u_new[1:Nx+1] = u[1:Nx+1] - CFL * (u[1:Nx+1] - u[0:Nx])}. Provide the Python code lines to set the ghost zone values of \lstinline{u_new} for each boundary condition type below:
\begin{enumerate}[label=\arabic*.]
    \item \textbf{Periodic boundaries:}
    The domain wraps around, so $u(0,t) = u(L,t)$.
    \item \textbf{Dirichlet boundaries:}
    The boundaries have fixed values, so $u(0,t)=0.5$ and $u(L,t)=0.0$.
    \item \textbf{Neumann boundaries:}
    The boundary gradients are set, so $u_x(0,t)=0.0$ and $u_x(L,t)=1.0$.
\end{enumerate}

\end{enumerate}

\end{document}
